\documentclass[a4paper,12pt]{article}
\usepackage{amsfonts}


\begin{document}

    %%%%%%%%%%%%%%%%%%%%%
    % Settings %
    %%%%%%%%%%%%%%%%%%%%%

    % Space before-after equations %
    % \setlength{\abovedisplayskip}{6pt}
    % \setlength{\belowdisplayskip}{6pt}
    \setlength{\abovedisplayshortskip}{6pt}
    \setlength{\belowdisplayshortskip}{24pt}
    
    % Space between paragraphs %
    \setlength\parskip{0.5\baselineskip}

    % New environment for definitions
    \newcounter{definition}[section]
    \newenvironment{definition}[1][]{\refstepcounter{definition}\par\medskip
        \noindent \textbf{Definition #1:} \rmfamily}{\medskip}
    
    %%%%%%%%%%%%%%%%%%%%%
    % Document %
    %%%%%%%%%%%%%%%%%%%%%
    \title{Thesis}
    \author{Daniel Travaglia}
    \date{\today}
    \maketitle

    \pagebreak

    % Here should go the index
    % ...
    \pagebreak
    
    % Let's divide the document into sections (i.e. chapters)
    \section{A gentle introduction to credit risk}

    % [TODO]
    % - Increasing comp. power, data collection, and regulation as main drivers for credit risk modelling
    % - 
    As you might have already inferred from the title, the aim of this section is to give an overview
    of the topic that is at the hearth of this paper: credit risk. In order to so however, rather than jumping 
    straight into formal definitions, which might create difficulties in grasping with the content, the reader 
    will be first provided with real-life examples characterized by the presence of such risk. Moving on, some
    formal definitions will be provided along with a description of the agents that usually take upon this risk 
    and how they act to assess it and quantify it with the purpose of taking proper actions to finally mitigate it. 
    Finally, the chapter ends with an introduction to credit risk modelling, as this topic will be  

    \subsection[]{Introduction}
            
        To start off this section, let me provide you with a few examples of situations that occur on a daily basis, which
        might however serve as a great starting point to understand the topic. 
    
        Imagine yourself receving a call from one of your friends that you have not heard for quite a while. 
        You can hear from his voice that he is really upset, and apparently, the reason for this is that he has to pay a fine 
        by the end of the day to avoid an additional charge, but he does not have the money to pay at the moment. 
        He tells you that the reason for this is that he had to cover fo huge expenses lately and he still has not received his monthly salary. 
        He then ask you, whether you would be kind enough to lend him the money, promising to repay you as soon as he receives his paid.
        What would you do? Would you lend him the money? What if one of your colleague at work asks you the same thing? Would you do it in this case? 
        Would it make a difference if it was someone you knew well? And if your colleague ask you to lend him money for a pizza without mentioning you when and 
        how is going to pay you back? 
    
        Usually, the type of questions that you ask yourself before getting into these agreements, which are also a simplified
        version of what financial istitutions ask themselves before lending, are the following: 

        \begin{enumerate}
            \item Who is borrowing?
            \item How much they are borrowing?
            \item When I will get it back?
            \item How much is there for me?
        \end{enumerate}

        Going back to the last example, imagine your colleague telling you that he will pay you back the week after and in addition, he is willing
        to take you out for lunch. In this case, he is paying you back more than what you lent him. Would you call this \textit{interest rate}? 
        What if instead he gives you his watch to keep until he pays you back? Now it feels a much more safer lending. Indeed, the watch
        in this case is usually called \textit{collateral} (or also \textit{security}).

        The procedure that has been just introduced is usually referred to as "\textbf{credit analysis}", which formally can be defined as the process through
        which the lender elaborates if he believes the counterparty is going to honor its obligations or not. This ultimately determine whether
        the agent will enter in that contract, along with the relative risk associated with it, or more precisely, the \textbf{credit risk}.

    \subsubsection{Credit risk}

        \begin{definition}[(Credit Risk)]
            the risk that a lender has to take into account due to the uncertainty related to the borrower either failing to repay a loan or to meet its obligations.
        \end{definition}

        In simple terms, this definition suggests that the lender should consider the possibility that a borrower will not be able to pay back the principal and the 
        interest rate according to the initial contractual agreement. He should then quantify this probability and based on the result, charge a coupon rate 
        (i.e. interest rate) to protect against this risk. The methods to derive the aformentioned probability are part of the broad area of \textbf{Credit Risk Modelling},
        which will be at the heart of the empirical analysis, and for this reason, it will be introduced later. 
        
        Ultimately, there are at least 3 additional points that is worth mentioning in the context of credit risk and that should be considered as a key takeaways from this introductory chapter: 
        
        \begin{enumerate}
            \item \textbf{Credit risk in every financial transaction}: although most of the times it might come natural to associate this type of risk to transactions that occur between a party and a financial istitution (e.g. mortgages, loans, credit cards, etc...), credit risk has a presence also at the company level, for instance, between companies and clients (e.g. paying invoices, insurance coverage, etc..)  
            \item \textbf{Risk assessment}: before granting new credit, as often it is the case in the business world, the bank undergo an assessment of the borrower that takes into consideration his credit history, the capabilities to repay back, the capital available, the loan's conditions and associated collateral. Such evaluation has the final goal of providing an accurate prior estimate of the credit risk, which will eventually tell whether the client should get or not the obligations, with an increasing interest rate for those who are perceived as riskier. In the case of bonds, this assessment is done by credit-rating agencies that assign a triple-A (i.e \textit{AAA}) for low-risk investments, all the way down to \textit{C} for high-risk investments. Although this process is tied to bonds only, our dataset provides ratings also for loans, perhaps as a result of internal procedures to assess credit risk.
            \item \textbf{Credit risk vs. other risks}: so far it was assumed that the risk associated to the lending was purely driven by the borrower characteristics. However, in a typical lending transaction, usually other types of risks factor in such as market risks (e.g. economic conditions, FX rates), country-specific risks (e.g. OECD country, emerging-markets country, etc..), operational risk, issuer risk, and so on... To keep things simple, and given the nature of the data for the empirical analysis, although it might represent a far too simplistic assumption, we will stick with it and only deal with a particular type of credit risk: the \textit{CCR (Counterparty Credit Risk)}
        \end{enumerate}
        
    \subsubsection{CCR - Counterparty Credit Risk}
        % Complete this section
        \begin{definition}[(Counterparty Credit Risk)]
            CCR can be defined as a measure of the \textbf{likelihood} that either one of the parties involved in a transaction \textbf{might default} on its contractual obligation.
        \end{definition}

        According to the definition of \textit{credit risk} provided above, there is a subtle but fundamental difference between the two type of risk, and in order to make it clear also to the reader, and example on loans is going
        to be provided. Imagine that a counterparty was able to obtain an incredibly high amount through a loan that was extended for whatever reason from a bank. The latter assessed whether the borrower would be able to 
        pay back the amount also within the date established with the lending contract. This process is also called "credit risk assessment" and takes into account the possibility that the borrower might not be able to fully repay the lender
        according to the contractual obligations, but do not exclude that the counterparty might cover for the remaining part in the future either. Hence, the bank here is exposed (in terms of risk) only to the portion of the money lended that
        believes are at risk. What if instead the banks start considering the chance that this counterparty might default on this loan such that he will never be able to pay back any of the amount received, neither in the present, nor in the future?
        Here the bank evaluates the risk at the counterparty level also known as "CCR" or "default risk" - the variable we wish to model.


    \pagebreak
    \section{Basel Regulatory Framework}

    The needs for a more comprehensive and better approach to risk management, particularly for counterparty credit risk,
    has emerged quite significantly after the financial crisis of 2007-2009. Since then, regulators have sharpened their 
    existing frameworks and applied more stringent controls to the stability of financial istitutions. 
    On this line of reasoning, this section provides an overview of how regulatory requirements in the context of counterparty credit
    risk have evolved over time, with a focus on the key regulatory framework in this environment: The Basel Accords. The aim is to make 
    the reader aware of the reasons to have such framework in place, along with a brief walk-through the history to see why, as of today, 
    the baking industry has become increasingly regulated, specially for what concern lending.
    \newline

    After the well-known and aformentioned 2007-2009 financial crisis event, there has been an unprecedented
    revision of the global framework regulating the financial sector, culminating in what is known today as the \textbf{Basel III} regulatory framework.
    Before reaching this result, the banking sector has experienced a relevant number of systemic crisis usually driven by various factors,
    including the miscalculation of risk represented by inadequate capital levels to carry out their business. Each of this crisis brought some contribution 
    and changes to the previous framework, adding up to the first version, which consisted of 30 pages, more than 1500 pages of guidelines relating to the 
    supervision of daily banking activities. 
    \newline

    Why was the Basel Committee ever needed? In order to answer this question, we have to go back to 1970s, when the Herstatt Bank collapsed and was put under liquidation 
    due to enormous trades on the foreign exchange market that did not go as planned. The license was withdrawn in the 1974, as losses have reached an amount equal to 10 times
    the liquidity of the bank. However, there is more to the story: US counterparties engaging in multiple transactions with Hersatt Bank released "Deutschmark" in exchange of dollars.
    These lenders never see their money, essentially because of time differences: US was still in morning trades when the bank was revoked its license. Although this is purely related to FX activities, 
    and cosequently involves also FX and market risk, this event highlighted the necessity to create a central forum for banking supervision concerning matters related also to other type of risks, 
    such as "credit risk". With the objective of enhancing the financial stability and quality of banking supervision, in 1974 multiple central banks gave raise to a centralized committee which later on 
    took the name of "Basel Committee on Banking Supervision". The latter expanded quite significantly and as of now, it includes 45 central banks worldwide.
    \newline

    The need for a regulatory framework for risk management was further strengthen during the 70s-80s period, when the surge in debt in the latin american countries combined with the raise of interest rates
    in US and Europe led the way to a series of critical debt restructuring efforts for many countries worldwide. Hence, the critical needs for a more clear and comprehensive framework to bring overall 
    stability in the banking system forced the Basel committee to issue guidelines on weighted approach to risk management. Such need was satisfied with the release of the Basel 1 framework in 1988, 
    when for the first time in history, banks were required to weigh the capital they held against the credit risk they took. 
    
    \subsection[]{Basel 1}
    In 1988, the first regulatory framework, which later took the name of "Basel I", was released by the Basel Committee. 
    The latter had the objective of setting up common international and shared regulations on capital adequacy supervision, something that then became the core aspect of all the activities that the Committee conduct. 
    The relevant changes that were applied are summarized in the following points:
    
        \begin{itemize}
            \item Istitutions that provide any type of lending services were required to classify assets into 5 different categories based on the underlying risk of holding such assets: from 0\% for most secured assets (e.g. cash) to 100\% for low-quality assets (e.g. private sector debt). Since then, "\textbf{RWA}" (risk-weighted assets) have been the keyword to describe this weighted approach towards assets.
            \item Definition of the so-called \textbf{Minimum Capital Requirement}: minimum ratio of capital to risk-weighted assets (RWA). The threshold was initially set at \textit{8\%}, equally spread between most absoring assets (i.e. \textbf{Tier-1 Capital}: examples are equity and retained earnings) and supplementary assets (\textbf{Tier-2 Capital}: examples are revaluated assets and subordinated debt). The latter includes assets that are more difficult to liquidate, and therefore less secure than the former. This level was introduced to ensure that financial istitutions had enough standalone capabilities to absord potential losses resulting from defaulting clients.
        \end{itemize}

    Despite the major alterations, the first Basel Regulation presented some shortcomings mainly related to the duration of the service, the market risk and, most importantly, the counterparty risk. Some of these issues were
    considered and partly assessed through amendments (such as the Market Risk Amendment). Nevertheless, the complexity introduced by some financial products (Credit default swaps, Complex derivatives, etc..) was drastically incrementing the risk taken by financial istitutions. 
    This situation required further adjustments with the regulation and shined a light on the ever-increasing importance of accurate methodologies to assess exposures to risk. 
     
    \subsection[]{Basel 2}
    The proposal for a replacement of the previous accord (i.e. Basel I) came in 1999, and was finally released in 2004 with the name \textbf{Basel II} .
    This comprised some of the aformentioned and much needed adjustments, and comprised the following 3 pillars:

        \begin{enumerate}
            \item \textbf{Minimum capital requirement}: developed and expanded on top of the standardized approach set out in the first approach and introduced a new capital tier that was meant to cover also for market risk (i.e. \textbf{Tier-3})
            \item \textbf{Market discipline}: effective period disclosure of risk exposures by financial istitutions to enable much more informed decisions
            \item \textbf{Supervisory review}: provided clear guidances on period assessments of an istitution's capital adequacy and granted intervention powers in case of presence of capital pressure 
        \end{enumerate}

    Despite the efforts to bring more stability to the financial environment, the accord required substantial time and effort before being ready and released. After the publication,
    most financial istitutions started taking full advantage of the subprime mortgages - lending money to low credit profile - thanks to their higher expected returns. Moreover, these institutions entered in the financial crisis 
    with way to much leverage and also took full advantage of the favourable economic conditions afterwards despite the regulatory environment that was set in place with Basel II,
    These events were part of a much broader series of misfunctionality (e.g. poor governance, bad risk management, combination of excessive credit growth and credit risk mispricing) 
    that served as key lessons for the Basel Committee to bring substantial updates to the baseline framework. Responding to these risk factors, the Committe agreed to design a reform
    package to outline new ways of handling credit and liquidity risk.

    \subsection[]{Basel 3}
    The agreement culminated with the publication in 2010 of what is now referred to as the \textbf{Basel III} framework. This was a very large package which aimed at improving
    on top of the 3 pillars of the previous accord and extend them to also other areas. The key changes can be summarized in the following points:
    
        \begin{enumerate}
            \item \textbf{Capital conservation buffer}: part of the common equity that, if ever breached, restricts payouts (e.g. dividends) to be used to meet the minimum common equity requirements
            \item \textbf{New leverage ratio}: based on loss-absorbing capital and total istitution's assets and takes into consideration also off-balance sheet exposures (regardless of RWA)
            \item \textbf{Liquidity Coverage Ratio (LCR)}: a minimum short-term liquidity ratio to provide banks with enough liquidity to cover for at least 30-days stress period.
            \item \textbf{Net Stable Funding Ratio (NSFR)}: a minimum long-term liquidity ratio to adress maturity mismatches over the balance sheet
        \end{enumerate}

    Despite the introduction of all these new regulations for the financial istitutions, the Basel Committee realized that the approach of "one fits all" is hardly to meet 
    the expectations outlined in the framwork, and therefore started with a series of reform to condition the requirements on the istitutions' size, complexity and systematic importance.
    This culminated once again in a series of on-going reform packages that is still happening today and should be published under the name: \textbf{Basel IV}.
    \newline

    % CONCLUSION -> FROM REGULATION -> TO HIRING IN CREDIT RISK -> RISK MODELLING -> CLOSE THE GAP WITH INFORMATION -> ASYMMETRIC IMPERFECTIONS
    
    \pagebreak
    \section{Credit Risk - Empirical Analysis}

    After an introduction to the theoretical foundations of credit risk and a walk-through the cornerstone events that had a significant impact for the regulation of the environment,
    it is time to take a more empirical approach and see how can data be put into valuable use in the context of risk exposures towards counterparties' loans granted by a financial istitution.
    However, before jumping straight into the analysis, an high-level overview of credit risk modelling as to what it is intended within financial istitutions will be provided, along with and the key terminologies in this area

    \subsection[]{Credit risk modelling}

    \subsubsection{Example}
    The motivation to develop internal credit risk models came, as we so before, from regulamentary requirements, but what was not stressed is the reason why this is such an important point
    for financial istitution. To better understand this, let us provide a simpliefied, but not too unrealistic example: imagine a bank is asked from a major tech company to provide a very huge
    loan in the size of \$5B. As already discussed, a credit analyst, who for ease the example we will call Benm will have to go through the request and see whether it is the case to grant or not this loan 
    based on the information that he is able to gather. 
    Let us assume that the analyst already knows that the CEO of this company has a great friendship relationship with the CFO of the bank, and, from recent studies, he discoveres that the specific sector 
    of the company has recently experiences a steady drop in sales and moreover, the bank-internal rating system suggests this company is on its way down to a subinvestment grade (i.e. classified as risky investment). 
    What should the analyst do at this point? The way to go usually boils down to 2 options at this point:

        \begin{enumerate}
            \item \textit{Reject} the deal based on the information he has gather on the company and the relative market
            \item \textit{Accept} the deal BUT, protect the bank against potential losses thanks to \textbf{credit risk management instruments} (such as credit derivatives) and transfer the risk to a third-party in exchange of a fee
        \end{enumerate}

    Hence, financial istitutions have also a way out: as individuals would do with health insurance, banks can protect themselves from the underlying exposures to risk when lending to particularly risky clients. 
    In particular, banks had already designed ways to get loan insurance in the past and usually comprises the whole bank's credit risk portfolio, and not only some critical positions. This brings directly to
    the building block of credit risk modelling within a financial istitution: the \textbf{expected loss}

    \subsubsection{Expected Loss}
    In order for banks to protect against this risk, they would like to know the \textit{cost} that arises from a particular client (a group of clients in the case of portfolio) 
    prior to granting the loan, so that they can charge a \textit{premium} on the interest rate accordingly and add up to a reserve (also calles \textit{expected loss reserve}) to cover for potential losses due to defaulting loans.
    What the bank is really interest here then is to know, \textit{on average}, the potential loss that might occur when lending money to a client. This is also known as \textbf{Expected Loss}, and it is characterized by the following 3 main building blocks of credit risk modelling:

        \begin{enumerate}
            \item \textbf{Probability of Default (PD)}: the \textit{likelihood} of a default in a defined time-horizon associated to a client. Note however that this is a general definition that may vary depending on the method used to estimate such probability. For instance, under the Basel II IRB framework, the PD per rating rate is defined as the average percentage percentage of obligators that will default over a one-year period, the time-horizon specified in the regulation for each asset class.
            \item \textbf{Exposure At Default (EAD)}: \textit{estimated outstanding amount} given that a client defaulted
            \item \textbf{Loss Given Default (LGD)}: \textit{proportion of EAD} that is expected not to be recovered in case of default.
        \end{enumerate}
    The model describing the expected loss is defined on top of a probability space ($\Omega,\Phi,\mathcal{P}$), where $\Omega$ consists of the \textit{sample space} defining all the possible events, 
    $\Phi$ is the \textit{$\sigma$-Algebra} containing all the measurable events of the sample space (such as the information on whether an obligator defaults or not) and lastly, 
    $\mathcal{P}$, the probability space, which attaches a probability to each measurable event.

    If we declare $\textbf{\textit{D}}_{i}$ as the event that the client '\textit{i}' defaults within a time-horizon and $\mathcal{P}(D_{i})$ as the proability associated to this event, then 
    the loss related to the obligator can be then defined as:
    
    \begin{equation}
        L_{i}=\iota*EAD_{i}*LGD_{i} \quad\mathrm{with}\quad  \iota = \mathrm{\textbf{1}}_{D_{i}}, \quad\mathcal{P}(D_{i}) = PD_{i}
    \end{equation}
    where $\iota$ is equal to the \textit{indicator function} of the underlying event \textit{D}. The variable takes value 1 when the 
    event occur (i.e. default) and 0 when it does not occur. Note that this is a Bernoulli R.V., and as such, the expectation is equal to the probability of the event to occur. In other words, it is equal to the \textit{Probability of Default (PD)}. 
    Before laying down the formula of \textit{Expected Loss}, another clarification should be provided:
    
    \begin{enumerate}
        \item To keep things simple, we will assume that the 3 constituents of the loss formula (i.e. PD, EAD, LGD) are independent among each other. Despite being a more than questionable statement and far from being true in general, this setting will allow us to state the most simple formula for the \textit{Expected loss}.
        \item To be able to simply define the expected loss over the whole portfolio of obligations, we also assume the joint distribution to be composed of independent and identically distributed R.V., each describing the expected loss of a specific obligator (\textit{i}).
    \end{enumerate}

    The formula of the expected loss of an obligator now comes very naturally as it is simply the expectation of the loss function defined above:

        \begin{equation}
            EL_{i}=\mathcal{P}(D_{i})*EAD_{i}*LGD_{i}
        \end{equation}
    
    where $\mathcal{P}(D_{i})$ represents the expectation of $\iota$.

    At this point, it is possible to identify the expected loss of a portfolio as the proportion of obligators that might default over a time-horizon,
    multiplied by the expected portion of the total expected exposures that is assumed not to be recovered if the default event occurs. More formally:

        \begin{equation}
            EL_{P}=\sum_{i=1}^{N} \mathcal{P}(D_{i})*EAD_{i}*LGD_{i}
        \end{equation}

    \subsubsection{Unexpected Loss}
    Note that the formula to calculate the expected losses aims at measuring the \textbf{expected} proportion of losses due to defaulting obligators based on historical default experiences, but 
    it does not consider the \textbf{unexpected} losses that might come up as a deviation from the average experienced losses of the past. The capital held to sustain
    losses should then consider both, and for the sake of completeness of this chapter, the formula will be provided here:

        \begin{equation}
            UL_{P}=\sum_{i=1}^{N} \sigma_{i}*\varphi_{i}
        \end{equation}
        where $\sigma_{i}$ is the individual standard deviation of credit losses for the obligator \textit{i}, while $\varphi_{i}$ indicates the correlation between 
        the credit losses of individual \textit{i} and all the other obligators in the portfolio.

    \subsubsection{Probability of Default}
    
    
    \pagebreak
    \subsection[]{Empirical work}
    

    \pagebreak
    \section{Blockchain and asymmetric information}

    \pagebreak
    \section{References}

    \begin{enumerate}
        \item Finance Unlocked (linK: )
        \item Investopedia - Credit Risk (link:)
        \item Introduction to credit risk modelling (Christian Bluhm)
        % \item Investopedia - Capital tiers (link: https://www.investopedia.com/ask/answers/043015/what-difference-between-tier-1-capital-and-tier-2-capital.asp)
        % \item Investopedia - Capital buffers (link: https://www.investopedia.com/terms/c/capital-buffer.asp)
        % \item Wikipedia - Debt Consolidation (link: https://en.wikipedia.org/wiki/Debt_consolidation)
        % \item Basel - History of Basel regulatory framework (link: https://www.bis.org/bcbs/history.htm?m=3%7C14%7C573%7C76)
        % \item Tier-2 Capital Definition https://www.investopedia.com/terms/t/tier2capital.asp

    \end{enumerate}

\end{document}