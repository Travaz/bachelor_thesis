\documentclass[a4paper,12pt]{article}
\begin{document}

    \newcounter{definition}[section]
    \newenvironment{definition}[1][]{\refstepcounter{definition}\par\medskip
        \noindent \textbf{Definition #1:} \rmfamily}{\medskip}
    
    % Some introduction
    \title{Thesis}
    \author{Daniel Travaglia}
    \date{\today}
    \maketitle

    \pagebreak

    % Here should go the index
    % ...
    \pagebreak
    
    % Let's divide the document into sections (i.e. chapters)
    \section{A gentle introduction to credit risk}

    To include in this section
    \begin{enumerate}
        \item What is it?
        \item Give me real life examples
        \item Why is it relevant and for who?
        \item Counterparty credit risk
    \end{enumerate}

    % [TODO]
    % - Increasing comp. power, data collection, and regulation as main drivers for credit risk modelling
    % - 
    As you might have already inferred from the title of this section, the aim here is to give an overview
    of the topic that is at the hearth of this paper: credit risk. In order to so however, rather than jumping 
    straight into formal definitions, which might create difficulties in grasping with the content, the reader 
    will be first provided with real-life examples characterized by the presence of such risk. Moving on, some
    formal definitions will be provided along with a description of the agents that usually take upon this risk 
    and how they act to assess it and quantify it with the purpose of taking proper actions to mitigate it. Finally, there ..

    \subsection[]{Introduction}
            
        To start off this section, let me provide you with a few examples of situations that occur on a daily basis, which
        might however serve as a great starting point to understand the topic. 
    
        Imagine yourself receving a call from one of your friends that you have not heard for quite a while. 
        You can hear from his voice that he is really upset, and apparently, the reason for this is that he has to pay a fine 
        by the end of the day to avoid an additional charge, but he does not have the money to pay at the moment, 
        due to huge expenses that he had to cover during that month and the fact that he still has not received his monthly salary. 
        He then ask you, whether you would be kind enough to lend him the money, promising to repay you as soon as he receives his paid.
        What would you do? Would you lend him the money? What if one of your colleague at work ask you the same thing? Would you do it in this case? 
        Would it make a difference if it was someone you knew well? And if your colleague ask you funds for a pizza but he does not mention when and 
        how is going to pay you back? 
    
        Usually, the type of questions that you ask yourself before getting into this type of agreements, which are also a simplified
        version of what financial istitutions ask themselves before giving any type of credit, are the following: 

        \begin{itemize}
            \item Who is borrowing?
            \item How much they are borrowing?
            \item When I will get it back?
            \item How much is there for me?
        \end{itemize}

        Going back to the last example, imagine your colleague telling you that he will pay you back the week after and in addition, he is willing
        to take you out for lunch. In this case, he is paying you back more than what you lent him. Would you call this \textit{interest rate}? 
        What if instead he gives you his watch to keep until he pays you back? This feels a much more safer lending, doesn't it? Indeed, the watch
        in this case is called \textit{security}.

        The procedure that has been just introduced is usually called "\textbf{credit analysis}", which formally can be defined as the process through
        which the lender elaborates if he believes the counterparty is going to honor its obligations or not. This ultimately determine whether
        the agent will enter in that contract or not, along with the relative risk associated with it, or more precisely, the \textbf{credit risk}.

    \subsubsection{Credit risk}

        Formally, credit risk can be defined as:
        \begin{definition}[(Credit risk)]
            the risk that a lender has to take into account due to the uncertainty related to the borrower failing to either repaying a loan or meet its obligations.
        \end{definition}

        In simple terms, this definition suggests that the lender should consider the possibility that a borrower will not be able to pay back the principal and the 
        interest rate according to the initial contractual agreement, quantify it and based on the result, charge a coupon rate (i.e. interest rate) to protect
        against this risk. 

        % - how is this probability quantified?
        % - bullshitta su dove possiamo trovare credit risk e provvedi a fare alcuni esempi
        % - default vs. credit risk

        

    


    % Old
    % The aim of this section is to provide an overview of the core topic of this paper: credit risk.
    % Being credit risk a vast and broad topic, which might result 
    % Instead of jumping straight into the relevant definitions and keywords that characterize this world,
    % the reader will be provided with examples concerning daily activities, with the hope that such provision
    % serves as a way to make the reader aware of the relevance of this type of risk in one's life.

    % Let me provide you with some examples: imagine yourself receiving a call from one of you colleague in a usual
    % split 
    
    \pagebreak
    \section{Basel Regulatory Framework}
    
    \pagebreak
    \section{Credit Risk Empirical Analysis}

    \begin{enumerate}
        \item 
    \end{enumerate}
    
    \pagebreak
    \section{Blockchain and asymmetric information}

    \pagebreak
    \section{References}

    \begin{enumerate}
        \item Finance Unlocked (linK: )
        \item Investopedia (link:)
        \item Introduction to credit risk modelling (Christian Bluhm)
    \end{enumerate}

\end{document}